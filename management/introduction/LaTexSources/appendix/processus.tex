\section{Processus des missions}

Cette annexe reprend et approfondit les éléments de réflexion sur la mission
des drones détaillés dans la partie deux. Après une brève introduction,
plusieurs schémas résumant les différentes configurations de réseau de
drones envisageables seront exposées avec le détail des missions.

\subsection{Présentation}

On peut imaginer réaliser la mission de secours donc selon plusieurs
scénarios.:\newline

\begin{itemize}
\item  Un seul drone dédié à larguer une charge utile parachutable. Cette situation suppose que la
  victime ait été préalablement mise au courant de quel tag il faut réaliser si on veut que le
     drone la repère.
\item  Un seul drone de secours qui fait le repérage et le largage. Cela
  implique qu’il recherche les victimes directement et non un tag. Les
  victimes peuvent être mobiles.
\item  Un drone de secours (pour le largage) et un drone de repérage. Le drone de repérage cherche
 et tague la victime et le drone de secours largue la charge utile parachutable sur le tag.
\item Un drone de secours (largage) et « j » drones de repérage.
\item  « i » drones de secours (largage) et 1 drone de repérage.
\item  « i » drones de secours (largage) et « j » drones de
    repérage.\newline

\end{itemize}

\subsection{Processus des missions}

\subsubsection{Drone de largage}

\begin{figure}[H]
    \begin{center}
        \includegraphics[scale=0.8]{body/images/1Drone.png}
    \end{center}
    \caption{1 unique drone de secours}
    \label{fig:1Largage}
\end{figure}


\subsubsection{Drone de largage et de repérage}

\begin{figure}[H]
    \begin{center}
        \includegraphics[scale=0.8]{body/images/1DroneAvecReperage.png}
    \end{center}
    \caption{1 unique drone de secours et de repérage}
    \label{fig:1DroneEtReperage}
\end{figure}


\subsubsection{Un drone de largage et un drone de repérage}

\begin{figure}[H]
    \begin{center}
        \includegraphics[scale=0.6]{body/images/1Drone1Reperage.png}
    \end{center}
    \caption{1 drone de secours et 1 drone de repérage}
    \label{fig:1Largage1Reperage}
\end{figure}


\subsubsection{Un drone de largage et un réseau de repérage}

\begin{figure}[H]
    \begin{center}
        \includegraphics[scale=0.6]{body/images/1DroneXReperage.png}
    \end{center}
    \caption{1 drone de secours et un réseau de repérage}
    \label{fig:1DroneXDrone}
\end{figure}


\subsubsection{Un réseau de largage et un drone de repérage}

\begin{figure}[H]
    \begin{center}
        \includegraphics[scale=0.6]{body/images/XDrone1Reperage.png}
    \end{center}
    \caption{1 réseau de largage et un drone de recherche}
    \label{fig:XDrone1Repererage}
\end{figure}


\subsubsection{Un réseau de largage et un réseau de repérage}

\begin{figure}[H]
    \begin{center}
        \includegraphics[scale=0.8]{body/images/XDroneXReperage.png}
    \end{center}
    \caption{Un réseau de largage et un réseau de repérage}
    \label{fig:XDroneXReperage}
\end{figure}

