\section*{Glossaire}
\addcontentsline{toc}{section}{Glossaire}
\begin{itemize}
  \item Drone\newline
Un drone est un aéronef, moyen de transport capable d'évoluer au sein de
l'atmosphère terrestre. Il ne possède pas de pilote à bord, d’où l’acronyme
anglais UAV (Unmanned Aerial Vehicle) ou plus récemment UAS (Unmanned
Aerial System). Les premiers drones étaient pilotés depuis le sol, mais
aujourd’hui l’on rencontre des drones de plus en plus autonomes, ne
nécessitant un pilote que pour les phases de décollage et d’atterrissage,
ou l'envoi d'ordres en vol. Il peut également être doté d'un pilotage
entièrement automatique. Par ailleurs, il peut être doté d'une prise de
décision opérationnelle autonome pour réagir en temps réel avec son
environnement. Il peut emporter une charge utile, et être destiné pour des
missions militaires ou civiles.\newline

  \item Système de drone\newline
On parle de plus ne plus de systèmes de drone. En effet, le drone (aéronef)
n'est qu'un élément du système qui comprend :
\begin{itemize}
  \item un ou plusieurs vecteurs aériens équipés de capteurs,
  \item une ou plusieurs bases au sol, de commande ou de recueil des
    informations,
  \item des liaisons de données (communications) entre le vecteur aérien et les
bases au sol.
\end{itemize}
De ce fait le terme drone désigne le plus souvent le vecteur aérien
uniquement.\newline

  \item Vecteur aérien\newline
Dans un système de drone, le vecteur aérien est le drone.\newline

  \item Réseau de drones\newline
On parle de réseaux de drones lorsque l’opérateur peut agir sur une entité
constituée d’un ensemble de drones. Des ordres individuels peuvent
cependant être envoyés.\newline

  \item Voilure plate\newline
Un drone à voilure plate est un drone dont le vol est assuré par une aile,
tel un avion.\newline

  \item Voilure tournante\newline
Un drone à voilure tournante est un drone dont le vol est assuré par des
hélices.\newline

  \item Système de secours\newline
Un ou plusieurs drones dont la fonction est le largage de la charge utile
parachutable.\newline

  \item Système de repérage\newline
Un ou plusieurs drones dont la fonction est le repérage de victimes dans la
zone sinistrée.\newline

  \item Victime\newline
Personne en difficulté dans la zone sinistrée.\newline

  \item Zone sinistrée\newline
Zone dans laquelle est susceptible de se trouver une ou plusieurs
victime(s).\newline

  \item Charge utile parachutable\newline
Charge larguée par le drone destinée à apporter des premiers secours aux
victimes. On parle aussi de kit de secours.\newline

  \item Tag\newline
Lors de la définition des sous-programmes, nous avons introduit la notion
de tag. Il s'agit du moyen utilisé par le drone ou le système de drone pour
repérer la zone d'intérêt. En fonction de la mission, de la technologie, le
tag peut aussi bien être un modèle physique que virtuel.\newline
S'il est physique, le tag peut être représenté par une forme géométrique
décidée au préalable par l'opérateur et connue des personnes à secourir. Il
pourra s'agir par exemple d'une croix au sol réalisée à l'aide de
vêtements. Le système de vision du drone repérera alors le tag, et ce
d'autant plus facilement que le contraste entre le tag et son support (le
sol, un toit) sera grand.\newline
S'il est virtuel, le tag peut être représenté par une forme plus complexe
connue du drone au départ de la mission : un oasis dans le désert, une
maison isolée, tout objet ou ensemble d'objet suffisamment unique dans son
environnement pour être détectable. Il peut également s'agir d'une
photographie de la zone d'intérêt. Le drone analysera alors les zones
visionnées par comparaison avec la photographie chargée en mémoire jusqu'à
obtenir un taux de similitude suffisamment important pour valider la
sélection de la zone d'intérêt.\newline
Une fois le tag repéré et validé, le drone de tagage transmet les
informations à l'opérateur afin qu'il les charge dans le drone de largage.
Il peut également envoyer les informations directement au drone de largage
si ce dernier fait parti du réseau de drone et qu'il est déjà en vol, en
attente de mission.\newline

\item Navigation autonome\newline
Déplacement d’un point A à un point B, définis par l’opérateur ou la
stratégie de navigation, selon un plan de vol calculé au lancement ou en
temps réel.\newline

\item Maillage\newline
Quadrillage spécifique, défini par l’opérateur, de la zone sinistrée que
doit suivre le drone de repérage, garantissant la couverture complète de la
zone sinistrée.\newline

\item Repérage\newline
Balayage de la zone sinistrée, éventuellement par maillage, afin de
localiser les victimes.\newline

\item Marquage\newline
Le marquage consiste à apposer un tag sur la victime identifiée, afin de la
rendre identifiable par le drone de largage.\newline

\item Ciblage\newline
Le ciblage constitue l’ensemble des actions à mettre en œuvre pour
localiser un tag défini lors du repérage. Cela désigne donc la stratégie de
navigation, l’identification de la cible et la sécurisation.\newline

\item Largage\newline
Action qui consiste à parachuter une charge utile aux coordonnées du
tag.\newline

  \item Drop zone\newline
Zone de largage définie grâce aux informations données par l’opérateur.
Dans les documents produits, l’abréviation DZ est parfois utilisée.
\end{itemize}
