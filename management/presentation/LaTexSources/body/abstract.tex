\section*{Résumé :}


Un document très cours donc, pour revenir sur le noyau de notre business. Pour
birèvement introduire le concept et présenter le cheminement de l'idée je
dirais que l'on va tout simplement: \newline

\begin{itemize}

    \item mettre en ligne notre système de trading
    \item y adjoindre des outils de développement pour créer un environnement
    à même d'attirer les cerveaux et de les faire travailler efficacement
    \item surplomber le tout d'une vitrine de visualisation
    \item enrober enfin cet éco-système d'une couche web-"marketing` pour en
    vanter les mérites et gérer les authentifications\newline

\end{itemize}

Le pari derrière cet agencement étant finalement de supposer tout d'abord qu'en
laissant vivre en live nos algorithmes de trading , ils vont mettre en oeuvre une
stratégie suffisamment efficace pour que des investisseurs ressentent l'intérêt
de payer pour la consulter. Un pari raisonnable à mon sens compte tenu des
sommes que ces investisseurs déboursent déjà dans les formations et autres
software d'analyse technique. Le marché existe, on innove dans la forme et je
considère que c'est une approche gagnante.\newline

L'autre pari est de supposer que le domaine passionnant de l'analyse
quantitative financière, couplé à un environnement de travail qui rox va attirer
une communauté compétente. Et si cela devait se confirmer, il se mettrait
clairement en place un puissant moteur de qualité et d'innovation, encore
propulsé par l'engouement montant pour l'open-source et l'open-data. Et encore une fois, 
considérant parmi tant d'autres les expériences de Quantopian, r-bloggers,
estimize, LuckySort, RecordedFuture ou QuanDL, c'est une supposition qui va
dans le sens des tendances atuelles.\newline


Partant de cette structure nous facturons donc la consultation des stratégies
algorithmiques, en mettant en place une quotation continue dépendante de leurs 
performances. En outre les développeurs ont l'opportunité de porter en live
leur travail, touchant ainsi une comission sur l'ensemble du revenu généré
par les clients qui ont "payé pour voir`.\newline
De cette manière nous favorisons la compétition, donc la performance, et
legitimons le prix des algorithmes puisqu'indexé sur leur revenu potentiel.
Enfin ce positionnement passif qui laisse l'initiaitve aux clients/développeurs
nous exonère du risque et de la juridiction inhérente à la gestion de
portefeuille.\newline
\newline


Quelques considérations annexes:\newline

\begin{itemize}

    \item La rétribution des développeurs pour leurs algorithmes lives est une
    zone tampon du business model. Mettre à disposition la structure de
    développement et de production est déjà une forme de rétribution. Le
    trading live a d'ailleurs une valeur significative mais on peut penser que ceux
    capables de produire des algorithmes ne sont pas ceux qui ont les moyens d'investir. 
    De plus le principe de nous rémunérer sur leur travail
    gracieusement mis en ligne pourrait être mal accueuilli. Il faut finalement
    évaluer si la valeur intellectuelle qu'ils apportent et la qualité
    attractive de la plateforme permettent ce modèle.\newline

    \item Je pense qu'assez logiquement il faut miser sur la coopération,
    surtout avec les outils web modernes qui le permettent.
    Le travail collaboratif des développeurs d'une part, mais aussi donner 
    la possibilité à l'investisseur de  discuter avec eux. Il a une expérience à
    apporter et ça lui donnera le sentiment de contrôler son investissement
    dans la visualisation de l'algo. J'en parle en annexe parce que ça ouvre la
    porte a beaucoup de choses et notamment des problèmes de privatisation. Il
    va vraiment falloir définir comment être sûr que les développeurs ne
    fassent pas fuiter n'importe comment les signaux.\newline

    \item L'académique: pourquoi pas imaginer des offres ou un professeur a un
    espace de travail auquel les élèves peuvent se connecter pour voir (c'est
    donc exactement le même principe, et il y a des choses très intéressantes à
    proposer avec l'interface originales développée en R avec Shiny). 
    Ou bien inversement, une offre où ce sont
    les élèves qui ont une sandbox et le professeur qui peut les consulter. J'y
    reviens souvent mais je pense toujours que dans notre optique
    l'environement académique peut nous amener des partenariats intéressants et
    de la ressource intellectuelle.\newline

    \item Dans le même ordre d'idée j'aimerais beaucoup explorer la relation
    que l'on peut avoir avec les coursera, udacity, edx, ou autre cours en
    ligne. Le système représente clairement une clé de l'avenir pédagogique et
    travailler avec le MIT ou Standford ne devrait pas être spécialement
    préjudiciable !\newline

    \item Autant l'offre de départ doit rester simple, autant il va falloir
    jalonner l'arrivée de nouveaux services au cours de la vie du soft. on a
    parlé de la clé usb (le grand retour) permettant d'importer ses données et de s'identifier,
    mais il y a des tas de possibilités avec notre base (organisation de
    concours, fractionnement de l'accès aux services, organisation de mini
    clubs d'investissement, sponsoring, application mobile, ...)\newline

\end{itemize}

On retrouve finalement pratiquement tout ce qu'on a évoqué depuis qu'on bosse
là-dessus et je trouve ça très bon signe. D'un point de vue développeurs nous
allons surfer sur ce qui va faire la puissance informatique des prochaines
années: puissance cloud, collaboration, open-source/data/cours, data
visualisation, ... D'un point de vue client investisseur on proposera une
version humanisée et cutting edges du trading algorithmique, laissant les
grandes institutions et les rêveurs se bouffer sur l'hyper fréquence et l'automatisation à
tout prix (littéralement).\newline

Enfin notre approche va nous permettre de capitaliser des infrastructures, des
connaissances, une communauté, un capital crédibilité, propre à faire émerger
d'énormes leviers d'action (hedge fund et club d'investissement bien sûr, mais
ppurquoi pas sortir du trading avec des champs applicatifs médicaux par exemple
! D'où l'intérêt de développer le squelette sur une base modulaire
d'ailleurs).\newline
\newline

\newpage

Le développement du projet se prolonge selon deux axes pas très originaux eux même divisés en
deux: \newline

\begin{itemize}

    \item La technique d'une part: Regroupant le développement des algorithmes et
l'amélioration du squelette applicatif et de son API. \newline

    \item Le business d'autre part avec d'un côté la formalisation du business model, de l'étude de
marché, de l'estimation des coûts, et de l'autre la mise en oeuvre de la
stratégie en allant notamment chercher les fonds, les partenariats et les
clients.\newline

\end{itemize}


Dans l'imédiat je vais ré-organiser selon ces nouvelles perspectives l'état de
l'art du projet. Avant de rejoindre Mahtieu sur le business model pendant qu'en
parallèle se développent les premiers algorithmes nécessaires aux tests. Et
prochain checkpoint samedi pour la suite de évènements !

