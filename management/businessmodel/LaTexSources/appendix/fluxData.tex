\section{Projet de gestion des flux}

Ce projet, évoqué en partie 2, n'a pas été développé mais tient une place
essentielle dans l'architecture du système et mérite donc sa place en
annexe.\newline

Son objectif est de fournir un module capable d'intégrer l'ensemble des
nombreuses données qui transitent au sein du drone. Il doit permettre de
garantir le traitement efficace de toutes les informations d'une part, et
de leur octroyer une plus-value intelligente d'autre part en les
enrichissant des axes temporels et spatiaux.\newline

\begin{figure}[H]
    \begin{center}
        \includegraphics[scale=0.6]{body/images/droneSkeleton.png}
    \end{center}
    \caption{Squelette fonctionnel du drone}
    \label{fig:squelette}
\end{figure}

Ce dispositif s'intègre dans la logique d'assister l'opérateur sur le plan
décisionnel, lui fournissant un état temps réel du fonctionnement de
l'appareil et de l'évolution de la mission. \newline

D'un point de vue technique, cette structure autorise le développement du
système complexe imaginé. En effet il est difficile d'imaginer que les
modules décisionnels soient efficaces sans reposer sur une production
synchronisée et fiable des données. On peut parler enfin de haute
disponibilité des données.\newline

En outre une telle maîtrise de fonctionnement renforce de manière évidente
la sécurité de l'appareil. En effet il devient possible d'intégrer de
manière cohérente des modules de contrôle et de diagnostique, permettant de
détecter en temps réel des défaillances. \newline

Une interface cloud devient de plus envisageable, autorisant par exemple
une communication optimisée du réseau de drones, qui deviendrait capable
d'accéder à tous moments aux états des autres systèmes.\newline

La structure du module est représentée figure x.

\begin{figure}[H]
    \begin{center}
        \includegraphics[scale=0.5]{body/images/schema_projet_DB.png}
    \end{center}
    \caption{Gestion du flux de données}
    \label{fig:fluxData}
\end{figure}


Notons qu'une telle architecture n'est pas présente sur les systèmes de vol
du même domaine et consitue donc une innovation.

L'un de ses autres avantages tient à la méthodologie de développement qu'il
induit. En effet il doit permettre de développer autour des modules plus
simplement et plus rapidement. 

Enfin une telle conception est générique, s'intéressant à un problème
récurrent au sein des systèmes, et est donc réutilisable.\newline

Pour conclure, ce concept pose les bases d'un développement d'intelligence
artificiel, temps réel et est surtout nécessaire à la cohérence des données
traitées.
